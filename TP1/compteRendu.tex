% !TEX encoding = IsoLatin
\documentclass{article}
\usepackage[french]{babel}
\usepackage[utf8]{inputenc}
\usepackage[T1]{fontenc}

\usepackage{amsmath}
\usepackage{amssymb}
\usepackage{amsfonts}
\usepackage{graphicx}

\usepackage[hmarginratio=1:1,top=32mm,columnsep=20pt]{geometry}
\usepackage{multirow}
\usepackage{multicol}
\usepackage{abstract}
\usepackage{fancyhdr}
\usepackage{float}

\usepackage[colorlinks=true,linkcolor=red,urlcolor=blue,filecolor=green]{hyperref}

\usepackage{dtklogos}
\usepackage{pbox}
\usepackage{caption}
\usepackage{mathtools}
\usepackage{listings}
\usepackage{mathrsfs}

\pagestyle{fancy}
\fancyhead{}
\fancyfoot{}
\fancyhead[C]{RO05 Automne 2014 - TP1}
\fancyfoot[RO, LE]{\thepage}

\newcommand{\bfx}{\mathbf(x)}
\newcommand{\transp}{^{\mathrm{t}}}

%-------------------------------------------------------------------------------

\title{RO05 TP1}

\author{Julien Amalfi, Benjamin Fradet}
\date{\today}

%-------------------------------------------------------------------------------

\begin{document}
\maketitle
\thispagestyle{fancy}

%-------------------------------------------------------------------------------

\begin{abstract}

    Au cours de ce TP, nous aurons l'occasion de générer des réalisations
    selon plusieurs lois: loi gaussienne, loi de Weibull, loi de Poisson, etc à
    partir de réalisations de nombres pseudo-aléatoires suivant
    $\mathcal{U}([0, 1])$.

\end{abstract}

%-------------------------------------------------------------------------------

\begin{multicols}{2}

\section{Introduction}\label{sec:intro}

Dans une première partie, nous définirons la fonction de répartition empirique.
Ensuite, nous générerons des réalisations de différentes lois grâce à
différentes techniques. Enfin, nous explorerons l'approximation de la loi
binomiale par la loi de Poisson.

%-------------------------------------------------------------------------------

\section{Exercice 1}\label{sec:ex1}

On a:

\begin{equation}
    \begin{split}
        \hat{F}_n(x) &= \frac{Card\{x_i \leq x; 1 \leq i \leq n\}}{n}, \forall x \in \mathbb{R} \\
                     &= \frac{1}{n} \times \sum_{i = 1}^{n} \mathbf{1}_{\{x_i \leq x\}}
    \end{split}
\end{equation}

En appliquant la loi des grands nombres, on trouve que:

\begin{equation}
    \begin{split}
        \hat{F}_n(x) &\to \mathbb{E}[\mathbf{1}_{\{x_i \leq x\}}]\text{, presque sûrement}\\
                     &\to 1 \times \mathbb{P}(\mathbf{1}_{\{x_i \leq x\}} = 1) + 0 \times \mathbb{P}(\mathbf{1}_{\{x_i \leq x\}} = 0)\\
                     &\to \mathbb{P}(\mathbf{1}_{\{x_i \leq x\}} = 1)\\
                     &\to \mathbb{P}(x_i \leq x)\\
                     &\to F(x)
    \end{split}
\end{equation}

Etant donné que les $x_i$ sont générés à partir des $ln(u_i)$, on peut supposer
que $F^{-1}(x) = ln(x)$ et donc, trivialement, que $F(x) = exp(x)$.

Si on génère $n = 200$ réalisations de la loi $\mathcal{U}([0, 1])$ et qu'on
leur applique la transformation $x_i = ln(u_i)$ et si l'on trace $F$ et
$\hat{F}_n$, on obtient:

\begin{figure}[H]
    \begin{center}
        \includegraphics[width=0.5\textwidth]{200n.png}
        \centering
        \captionsetup{justification=centering}
        \caption{\label{fig:200n}Comparaison entre fonction de répartition théorique et empirique pour $n = 200$}
    \end{center}
\end{figure}

On peut voir en rouge la fonction de répartition empirique et en bleu la
fonction de répartition théorique.

Si on répète l'expérience, en reprenant les mêmes codes couleurs, avec
$n = 1000$:

\begin{figure}[H]
    \begin{center}
        \includegraphics[width=0.5\textwidth]{1000n.png}
        \centering
        \captionsetup{justification=centering}
        \caption{\label{fig:1000n}Comparaison entre fonction de répartition théorique et empirique pour $n = 1000$}
    \end{center}
\end{figure}

Enfin, avec $n = 10000$:

\begin{figure}[H]
    \begin{center}
        \includegraphics[width=0.5\textwidth]{10000n.png}
        \centering
        \captionsetup{justification=centering}
        \caption{\label{fig:10000n}Comparaison entre fonction de répartition théorique et empirique pour $n = 10000$}
    \end{center}
\end{figure}

On voit que plus $n$ grandit, plus la fonction de répartition empirique se
rapproche à la fonction de répartition théorique, conséquence directe de la
définition de $\hat{F}_n$ (à élaborer).

%-------------------------------------------------------------------------------

\section{Exercice 2}\label{sec:ex2}

\subsection{$X_1 \sim \mathcal{P}(3)$}\label{subsec:ex21}

Pour la génération de réalisations de la loi de Poisson, on utilise le résultat
suivant:

Si $(E_i)_{i \in \mathbb{N}^*} \sim \mathcal{E}(1)$ alors
\begin{equation}
    X = inf\{n \geq 0 \,|\, \sum_{i = 1}^{n + 1} E_i > \lambda\} \sim
        \mathcal{P}(\lambda)
\end{equation}
Donc, si $(U_i)_{i \in \mathbb{N}^*} \sim \mathcal{U}([0, 1])$ alors
\begin{equation}
    X = inf\{n \geq 0 \,|\, \prod_{i = 1}^{n + 1} U_i < \mathrm{e}^{-\lambda}\}
        \sim \mathcal{P}(\lambda)
\end{equation}

On génère donc des réalisations de $\mathcal{U}([0, 1])$ que l'on multiplie
entre elles jusqu'à ce que le résultat soit inférieur à $\mathrm{e}^{-\lambda}$
et la réalisation de la loi de Poisson sera le nombre de multiplications
effectué.

On applique la même technique que dans ~\ref{sec:ex1} pour calculer la fonction
de répartition empirique que l'on trace en rouge. De plus, on trace aussi la
fonction de répartition empirique obtenue à partir de réalisations faites
directement par scilab à l'aide de \texttt{grand} en bleu. Enfin, on trace la
fonction de répartition théorique en noir.

\begin{figure}[H]
    \begin{center}
        \includegraphics[width=0.5\textwidth]{poisson.png}
        \centering
        \captionsetup{justification=centering}
        \caption{\label{fig:poisson}Comparaison entre fonctions de répartition théorique et empiriques de la loi de Poisson $\lambda = 3$ à partir de $n = 500$ réalisations}
    \end{center}
\end{figure}

On voit que la fonction de répartition empirique calculée à partir des
réalisations faites par Scilab semble plus se rapprocher de la fonction de
répartition théorique. Ceci est peut-être dû au fait que Scilab utilise une
technique plus adaptée pour générer des réalisations de la loi de Poisson que
l'approximation par lois exponentielles.

\subsection{$X_2 \in \{-1, 0, 1\} \,|\, \mathbb{P}(X_2 = -1) = \mathbb{P}(X_2 = 0) = \mathbb{P}(X_2 = 1)$}\label{subsec:ex22}

Pour cette loi, on génère simplement des réalisations de $\mathcal{U}([0, 1])$
et on génère:

\begin{itemize}
    \item $x_2 = -1 \text{ si } u_i \leq \frac{1}{3}$
    \item $x_2 = 0  \text{ si } \frac{1}{3} \leq u_i \leq \frac{2}{3}$
    \item $x_2 = 1  \text{ sinon}$
\end{itemize}

La fonction de répartition théorique, quant à elle, est donnée par:

\begin{itemize}
    \item $\mathbb{P}(X_2 < -1) = 0$
    \item $\mathbb{P}(X_2 \leq -1) = \mathbb{P}(X_2 = -1) = \frac{1}{3}$
    \item $\mathbb{P}(X_2 \leq 0) = \mathbb{P}(X_2 \leq -1) +
        \mathbb{P}(X_2 = 0) = \frac{1}{3} + \frac{1}{3} = \frac{2}{3}$
    \item $\mathbb{P}(X_2 \leq 1) = \mathbb{P}(X_2 \leq 0) + \mathbb{P}(X_2 = 1)
        = \frac{2}{3} + \frac{1}{3} = 1$
\end{itemize}

Etant donné qu'il n'existe pas de fonction Scilab pour générer des réalisations
de cette loi, on visualise uniquement la fonction de répartition empirique
calculée à partir de nos réalisations, en pointillés rouges, ainsi que la
fonction de répartition théorique, en noir.

\begin{figure}[H]
    \begin{center}
        \includegraphics[width=0.5\textwidth]{oneThird.png}
        \centering
        \captionsetup{justification=centering}
        \caption{\label{fig:oneThird}Comparaison entre fonctions de répartition théorique et empirique d'une loi de probabilité donnée à partir de $n = 500$ réalisations}
    \end{center}
\end{figure}

On voit ces deux fonctions de répartitions sont très semblables.

\subsection{$X_3 \sim \mathcal{N}(2, 4)$}\label{subsec:ex23}

Pour ce qui est de la loi nomale, on utilise la méthode de Box-Müller qui, étant
données deux variables alétoires $U_1$ et $U_2$ suivant $\mathcal{U}([0, 1])$ et
en définissant $R = \sqrt{-2 ln(U_1)}$ et $\Theta = 2 \pi U_2$, permet de
générer deux variables aléatoires suivant la loi normale centrée réduite:

\begin{itemize}
    \item $X_1 = R\cos(\Theta)$
    \item $X_2 = R\sin(\Theta)$
\end{itemize}

On génère donc des réalisations de la loi normale centrée-réduite grâce à cette
méthode. Ensuite, on "dénormalise" ces réalisations en les multipliant par
l'écart type $\sigma = 4$ et on leur rajoute la moyenne $\mu = 2$, pour obtenir
des réalisations suivant $\mathcal{N}(2, 4)$.

En plus de la fonction de répartition empirique obtenue à partir de ces
réalisations, affichée en pointillés rouge ci-après, on calcule la fonction de
répartition empirique pour des réalisations générées à l'aide de la fonction
Scilab \texttt{grand}, en pointillés bleus. Finalement, on affiche aussi la
fonction de répartition de la loi normale à l'aide de la fonction
\texttt{cdfnor}, en noir.

\begin{figure}[H]
    \begin{center}
        \includegraphics[width=0.5\textwidth]{gaussienne.png}
        \centering
        \captionsetup{justification=centering}
        \caption{\label{fig:gaussienne}Comparaison entre fonctions de répartition théorique et empiriques d'une loi normale $\mu = 2$ et $\sigma = 4$ à partir de $n = 500$ réalisations}
    \end{center}
\end{figure}

On voit que nos trois fonctions de répartitions sont quasiment confondues, ce
qui nous permet de dire que la méthode de Box-Müller semble très bien
fonctionnée.

\subsection{$X_4 \sim Weibull(3, 3)$}\label{subsec:ex24}

En ce qui concerne la loi de Weibull, on procède par la méthode d'inversion de
la fonction de répartition:

\begin{equation}
    \begin{multlined}
        F(x) = 1 - \mathrm{e}^{-\left({\frac{x}{\lambda}}\right)^k} = y \\
        1 - y = \mathrm{e}^{-\left({\frac{x}{\lambda}}\right)^k} \\
        ln(1 - y) = -\left(\frac{x}{\lambda}\right)^k \\
        (-ln(1 - y))^{\frac{1}{k}} = \frac{x}{\lambda} \\
        x = \lambda (-ln(1 - y))^{\frac{1}{k}}
    \end{multlined}
\end{equation}

On génère donc des réalisations $u_i$ de $U$ suivant $\mathcal{U}([0, 1])$ et on
leur applique la transformation suivante: $\lambda (-ln(1 - u_i))^{frac{1}{k}}$
pour obtenir des réalisations de la loi de Weibull de paramètres $\lambda$ et
$k$.

Etant donné qu'il n'existe pas de fonction Scilab pour générer des réalisations
suivant la loi de Weibull, on trace uniquement notre fonction de répartition
empirique, en rouge pointillés, ainsi que la fonction de répartition théorique,
en noir.

\begin{figure}[H]
    \begin{center}
        \includegraphics[width=0.5\textwidth]{weibull.png}
        \centering
        \captionsetup{justification=centering}
        \caption{\label{fig:gaussienne}Comparaison entre fonctions de répartition théorique et empirique d'une loi de Weibull $\lambda = 3$ et $k = 3$ à partir de $n = 500$ réalisations}
    \end{center}
\end{figure}

On voit que la technique de génération par inversion semble être une bonne
méthode.

%-------------------------------------------------------------------------------

\section{Exercice 3}\label{sec:ex3}

\subsection{Approximation de $\mathcal{B}(400, 0.0037)$ par $\mathcal{P}(1.48)$}\label{subsec:ex31}

On génère des réalisations suivant ces deux lois grâce à la fonction
\texttt{grand}. On affiche ensuite un histogramme des réalisations, en rouge les
réalisations la loi de Poisson et en noir les réalisations de la loi binomiale.

\begin{figure}[H]
    \begin{center}
        \includegraphics[width=0.5\textwidth]{binPoi1.png}
        \centering
        \captionsetup{justification=centering}
        \caption{\label{fig:binPoi1}Comparaison entre les distributions d'une loi binomiale $n = 400$ et $p = 0.0037$ et d'une loi de Poisson $\lambda = 1.48$}
    \end{center}
\end{figure}

On voit que l'approximation semble être correcte lorsque $\lambda = n \times p$.

\subsection{Approximation de $\mathcal{B}(400, 0.039)$ par $\mathcal{P}(15.6)$}\label{subsec:ex31}

On procède de la même manière pour ces distributions, et on obtient:

\begin{figure}[H]
    \begin{center}
        \includegraphics[width=0.5\textwidth]{binPoi2.png}
        \centering
        \captionsetup{justification=centering}
        \caption{\label{fig:binPoi2}Comparaison entre les distributions d'une loi binomiale $n = 400$ et $p = 0.039$ et d'une loi de Poisson $\lambda = 15.6$}
    \end{center}
\end{figure}

Là aussi, l'approximation semble être correcte.

%-------------------------------------------------------------------------------

\section{Conclusion}\label{sec:conclu}

ex3 -> avec n plus grand

%-------------------------------------------------------------------------------

%-------------------------------------------------------------------------------

\end{multicols}
\end{document}
