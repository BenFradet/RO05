% !TEX encoding = IsoLatin
\documentclass{article}
\usepackage[french]{babel}
\usepackage[utf8]{inputenc}
\usepackage[T1]{fontenc}

\usepackage{amsmath}
\usepackage{amssymb}
\usepackage{amsfonts}
\usepackage{graphicx}

\usepackage[hmarginratio=1:1,top=32mm,columnsep=20pt]{geometry}
\usepackage{multirow}
\usepackage{multicol}
\usepackage{abstract}
\usepackage{fancyhdr}
\usepackage{float}

\usepackage[colorlinks=true,linkcolor=red,urlcolor=blue,filecolor=green]{hyperref}

\usepackage{dtklogos}
\usepackage{pbox}
\usepackage{caption}
\usepackage{mathtools}
\usepackage{listings}
\usepackage{mathrsfs}

\pagestyle{fancy}
\fancyhead{}
\fancyfoot{}
\fancyhead[C]{RO05 Automne 2014 - TP2}
\fancyfoot[RO, LE]{\thepage}

\newcommand{\bfx}{\mathbf(x)}
\newcommand{\transp}{^{\mathrm{t}}}

%-------------------------------------------------------------------------------

\title{RO05 TP2}

\author{Julien Amalfi, Benjamin Fradet}
\date{\today}

%-------------------------------------------------------------------------------

\begin{document}
\maketitle
\thispagestyle{fancy}

%-------------------------------------------------------------------------------

\begin{abstract}

    Comme lors du premier TP, nous etudierons la generation de nombre
    aleatoires, cette fois-ci plus particulierement pour les lois de Weibull et
    de Cauchy.

\end{abstract}

%-------------------------------------------------------------------------------

\begin{multicols}{2}

\section{Introduction}\label{sec:intro}

Dans une premiere partie, nous etudierons la loi de Weibull en generant des
realisations par la methode d'inversion, puis nous utiliserons ces realistions
pour approximer une integrale. Ensuite, nous simulerons des realisations d'une
loi de probabilite atypique. Enfin, nous nous pencherons sur la loi de Cauchy
et pourquoi elle ne verifie pas la loi forte des grands nombres.

%-------------------------------------------------------------------------------

\section{Exercice 1}\label{sec:ex1}

\subsection{Generation de $n = 1000$ realisations de $X \sim Weibull(4, 1)$}\label{subsec:ex11}

On procede par inversion:

\begin{equation}
    \begin{multlined}
        F(x) = 1 - \mathrm{e}^{-\left({\frac{x}{\lambda}}\right)^k} = y \\
        1 - y = \mathrm{e}^{-\left({\frac{x}{\lambda}}\right)^k} \\
        ln(1 - y) = -\left(\frac{x}{\lambda}\right)^k \\
        (-ln(1 - y))^{\frac{1}{k}} = \frac{x}{\lambda} \\
        x = \lambda (-ln(1 - y))^{\frac{1}{k}}
    \end{multlined}
\end{equation}

On génère donc des réalisations $u_i$ de $U$ suivant $\mathcal{U}([0, 1])$ et on
leur applique la transformation suivante: $\lambda (-ln(1 - u_i))^{\frac{1}{k}}$
pour obtenir des réalisations de la loi de Weibull de paramètres $\lambda$ et
$k$.

On trace donc la fonction de repartition empirique, a l'aide de ces realisations
ainsi que la fonction de repartition theorique.

\begin{figure}[H]
    \begin{center}
        \includegraphics[width=0.5\textwidth]{weibull.png}
        \centering
        \captionsetup{justification=centering}
        \caption{\label{fig:weibull}Comparaison entre fonctions de répartition théorique et empirique de la loi de Weibull $\alpha = 4$ $\beta = 1$ à partir de $n = 1000$ réalisations}
    \end{center}
\end{figure}

On peut voir la fonction de repartition empirique en rouge pointilles ainsi que
la fonction de repartition theorique en noir.

\subsection{Calcul de la densite de la distribution de Weibull}\label{subsec:ex12}

On derive tout simplement la fonction de repartition:

\begin{equation}
    \begin{split}
        f(x) &= F'(x) \\
             &= \frac{\alpha}{\beta}\left(\frac{x}{\beta}\right)^{\alpha - 1}\mathrm{e}^{-\left(\frac{x}{\beta}\right)^{\alpha}} \\
             &= 4 x^3 \mathrm{e}^{-x^4}
    \end{split}
\end{equation}

\subsection{Calcul de $I = \int_0^{+\infty} x^6 \mathrm{e}^{-x^4} dx$}\label{subsec:ex13}

On s'aide de la methode de Monte-Carlo, qui nous dit que, si on pose
$g(x) = \frac{x^2}{4}$ et $f(x) = 4 x^3 \mathrm{e}^{-x^4}$, on a:

\begin{equation}
    \begin{split}
        I &= \int_0^{+\infty} x^6 \mathrm{e}^{-x^4} dx \\
          &= \int_0^{+\infty} f(x) g(x) dx \\
          &= \mathbb{E}(g(X))
    \end{split}
\end{equation}

Ici, la loi forte des grands nombres nous permet d'affirmer que:

\begin{equation}
    \frac{1}{n} \sum_{i = 1}^n g(X_i) \to \mathbb{E}(g(X)) \text{, presque surement}
\end{equation}

On applique donc $g$ a nos realisations et on en fait la moyenne pour obtenir
notre resultat. On obtient 0.219 pour $n = 100$ et 0.215 pour $n = 1000$.

%-------------------------------------------------------------------------------

\section{Exercice 2}\label{sec:ex2}

\subsection{$f$ densite}\label{subsec:ex21}

Pour montrer que $f$ est une densite on l'integre sur $]-\infty, +\infty[$ et on
verifie que l'on obtient bien 1:

\begin{equation}
    \begin{split}
        \int_{-\infty}^{+\infty} f(x) dx
        &= \int_{-\infty}^{+\infty} \frac{1}{\pi \sqrt{1 - x^2}} 1_{]-1, 1[}(x) dx \\
        &= \int_{-1}^{+1} \frac{1}{\pi \sqrt{1 - x^2}} dx \\
        &= \frac{1}{\pi} \int_{-1}^{+1} \frac{1}{\sqrt{1 - x^2}} dx \\
        &= \frac{1}{\pi} [arcsin(x)]_{-1}^{+1} \\
        &= \frac{1}{\pi} \left(\frac{\pi}{2} - \left(\frac{-\pi}{2}\right)\right) \\
        &= 1
    \end{split}
\end{equation}

On peut donc en deduire que $f$ est une densite.

%-------------------------------------------------------------------------------

\section{Exercice 3}\label{sec:ex3}

\subsection{Approximation de $\mathcal{B}(400, 0.0037)$ par $\mathcal{P}(1.48)$}\label{subsec:ex31}

On génère des réalisations suivant ces deux lois grâce à la fonction
\texttt{grand}. On affiche ensuite un histogramme des réalisations, en rouge les
réalisations la loi de Poisson et en noir les réalisations de la loi binomiale.

On voit que l'approximation semble être correcte lorsque $\lambda = n \times p$.

\subsection{Approximation de $\mathcal{B}(400, 0.039)$ par $\mathcal{P}(15.6)$}\label{subsec:ex31}

On procède de la même manière pour ces distributions, et on obtient:

Là aussi, l'approximation semble être correcte.

%-------------------------------------------------------------------------------

\section{Conclusion}\label{sec:conclu}

En conclusion, ce TP nous aura permis d'appréhender différentes techniques de
génération de nombres alétoires.

Pour aller plus loin, on aurait pu essayer de générer des réalisations à l'aide
de la méthode de rejet que nous n'avons pas utilisée ici.
On aurait pu aussi examiner le comportement de l'approximation de la loi
binomiale par la loi de Poisson pour $n \gg 400$ dans le but de voir si
l'approximation se révèle meilleure.

%-------------------------------------------------------------------------------

\end{multicols}
\end{document}
