% !TEX encoding = IsoLatin
\documentclass{article}
\usepackage[french]{babel}
\usepackage[utf8]{inputenc}
\usepackage[T1]{fontenc}

\usepackage{amsmath}
\usepackage{amssymb}
\usepackage{amsfonts}
\usepackage{graphicx}

\usepackage[hmarginratio=1:1,top=32mm,columnsep=20pt]{geometry}
\usepackage{multirow}
\usepackage{multicol}
\usepackage{abstract}
\usepackage{fancyhdr}
\usepackage{float}

\usepackage[colorlinks=true,linkcolor=red,urlcolor=blue,filecolor=green]{hyperref}

\usepackage{dtklogos}
\usepackage{pbox}
\usepackage{caption}
\usepackage{mathtools}
\usepackage{listings}
\usepackage{mathrsfs}

\pagestyle{fancy}
\fancyhead{}
\fancyfoot{}
\fancyhead[C]{RO05 Automne 2014 - TP2}
\fancyfoot[RO, LE]{\thepage}

\newcommand{\bfx}{\mathbf(x)}
\newcommand{\transp}{^{\mathrm{t}}}

%-------------------------------------------------------------------------------

\title{RO05 TP2}

\author{Julien Amalfi, Benjamin Fradet}
\date{\today}

%-------------------------------------------------------------------------------

\begin{document}
\maketitle
\thispagestyle{fancy}

%-------------------------------------------------------------------------------

\begin{abstract}

    Comme lors du premier TP, nous étudierons la génération de nombre
    aléatoires, cette fois-ci plus particulièrement pour les lois de Weibull et
    de Cauchy.

\end{abstract}

%-------------------------------------------------------------------------------

\begin{multicols}{2}

\section{Introduction}\label{sec:intro}

Dans une première partie, nous étudierons la loi de Weibull en générant des
réalisations par la méthode d'inversion, puis nous utiliserons ces réalistions
pour approximer une intégrale par la méthode de Monte-Carlo. Ensuite, nous
simulerons des réalisations d'une loi de probabilité atypique. Enfin, nous nous
pencherons sur la loi de Cauchy et pourquoi elle ne vérifie pas la loi forte des
grands nombres.

%-------------------------------------------------------------------------------

\section{Exercice 1}\label{sec:ex1}

\subsection{Génération de $n = 1000$ réalisations de $X \sim Weibull(4, 1)$}\label{subsec:ex11}

On procède par inversion:

\begin{equation}
    \begin{multlined}
        F(x) = 1 - \mathrm{e}^{-\left({\frac{x}{\lambda}}\right)^k} = y \\
        1 - y = \mathrm{e}^{-\left({\frac{x}{\lambda}}\right)^k} \\
        ln(1 - y) = -\left(\frac{x}{\lambda}\right)^k \\
        (-ln(1 - y))^{\frac{1}{k}} = \frac{x}{\lambda} \\
        x = \lambda (-ln(1 - y))^{\frac{1}{k}}
    \end{multlined}
\end{equation}

On génère donc des réalisations $u_i$ de $U \sim \mathcal{U}([0, 1])$ et on
leur applique la transformation suivante: $\lambda (-ln(1 - u_i))^{\frac{1}{k}}$
pour obtenir des réalisations de la loi de Weibull de paramètres $\lambda$ et
$k$.

On trace donc la fonction de répartition empirique, à l'aide de ces
réalisations, ainsi que la fonction de répartition théorique.

\begin{figure}[H]
    \begin{center}
        \includegraphics[width=0.5\textwidth]{weibull.png}
        \centering
        \captionsetup{justification=centering}
        \caption{\label{fig:weibull}Comparaison entre fonctions de répartition théorique et empirique de la loi de Weibull $\alpha = 4$ $\beta = 1$ à partir de $n = 1000$ réalisations}
    \end{center}
\end{figure}

On peut voir la fonction de répartition empirique en pointillés rouges ainsi que
la fonction de répartition théorique en noir.

\subsection{Calcul de la densité de la distribution de Weibull}\label{subsec:ex12}

On dérive tout simplement la fonction de répartition:

\begin{equation}
    \begin{split}
        f(x) &= F'(x) \\
             &= \frac{\alpha}{\beta}\left(\frac{x}{\beta}\right)^{\alpha - 1}\mathrm{e}^{-\left(\frac{x}{\beta}\right)^{\alpha}} \\
             &= 4 x^3 \mathrm{e}^{-x^4}
    \end{split}
\end{equation}

\subsection{Calcul de $I = \int_0^{+\infty} x^6 \mathrm{e}^{-x^4} dx$}\label{subsec:ex13}

On s'aide de la méthode de Monte-Carlo, qui nous dit que, si on pose
$g(x) = \frac{x^2}{4}$ et $f(x) = 4 x^3 \mathrm{e}^{-x^4}$, on a:

\begin{equation}
    \begin{split}
        I &= \int_0^{+\infty} x^6 \mathrm{e}^{-x^4} dx \\
          &= \int_0^{+\infty} f(x) g(x) dx \\
          &= \mathbb{E}(g(X))
    \end{split}
\end{equation}

$X$ étant une variable aléatoire ayant pour densité $f$.

Ici, la loi forte des grands nombres nous permet d'affirmer que:

\begin{equation}
    \frac{1}{n} \sum_{i = 1}^n g(X_i) \to \mathbb{E}(g(X)) \text{, presque sûrement}
\end{equation}

On applique donc $g$ à nos réalisations et on en fait la moyenne pour obtenir
notre résultat. On obtient 0.219 pour $n = 100$ et 0.215 pour $n = 1000$.

De plus, si on trace l'évolution de cette intégrale en fonction du nombre de
réalisations générées, on voit qu'elle semble converger:

\begin{figure}[H]
    \begin{center}
        \includegraphics[width=0.5\textwidth]{intWeibull.png}
        \centering
        \captionsetup{justification=centering}
        \caption{\label{fig:intWeibull}Convergence de l'intégrale en fonction du nombre de réalisations générées pour $n_{max} = 1000$ réalisations}
    \end{center}
\end{figure}

%-------------------------------------------------------------------------------

\section{Exercice 2}\label{sec:ex2}

\subsection{$f$ densité}\label{subsec:ex21}

Pour montrer que $f$ est une densité on l'intègre sur $]-\infty, +\infty[$ et on
vérifie que l'on obtient bien 1:

\begin{equation}
    \begin{split}
        \int_{-\infty}^{+\infty} f(x) dx
        &= \int_{-\infty}^{+\infty} \frac{1}{\pi \sqrt{1 - x^2}} 1_{]-1, 1[}(x) dx \\
        &= \int_{-1}^{+1} \frac{1}{\pi \sqrt{1 - x^2}} dx \\
        &= \frac{1}{\pi} \int_{-1}^{+1} \frac{1}{\sqrt{1 - x^2}} dx \\
        &= \frac{1}{\pi} [arcsin(x)]_{-1}^{+1} \\
        &= \frac{1}{\pi} \left(\frac{\pi}{2} - \left(\frac{-\pi}{2}\right)\right) \\
        &= 1
    \end{split}
\end{equation}

On peut donc en déduire que $f$ est une densité.

\subsection{Génération de $n = 1000$ réalisations de $X$}\label{subsec:ex22}

Etant donné que $f$ est une densité, on peut l'intégrer et obtenir la fonction
de répartition $F$:

\begin{equation}
    \begin{split}
        F(x)
        &= \int_{-\infty}^{x} f(y) dy \\
        &= \int_{-\infty}^{x} \frac{1}{\pi \sqrt{1 - y^2}} 1_{]-1, 1[}(y) dy \\
        &= \int_{-1}^{x} \frac{1}{\pi \sqrt{1 - y^2}} dy \\
        &= \frac{1}{\pi} \int_{-1}^{x} \frac{1}{\sqrt{1 - y^2}} dy \\
        &= \frac{1}{\pi} [arcsin(y)]_{-1}^{x} \\
        &= \frac{1}{\pi} \left(arcsin(x) + \frac{\pi}{2}\right) 1_{]-1, 1[}(x)
    \end{split}
\end{equation}

On procède par inversion de cette fonction de répartition:

\begin{equation}
    \begin{split}
        F(x) &= \frac{arcsin(x)}{\pi} + \frac{1}{2} = y \\
             & \left(y - \frac{1}{2}\right) \pi = arcsin(x) \\
             & x = sin\left(\left(y - \frac{1}{2}\right) \pi \right)
    \end{split}
\end{equation}

On génère donc $n = 1000$ réalisations $u_i$ de $U \sim \mathcal{U}([0, 1])$ et
on leur applique la transformation
$sin\left(\left(u_i - \frac{1}{2}\right) \pi \right)$ pour obtenir des
réalisations de $X$.

\subsection{Tracé des fonctions de répartitions}\label{subsec:ex23}

Si on trace la fonction de répartition empirique en pointillés rouges et la
fonction de répartition théorique en noir, on obtient:

\begin{figure}[H]
    \begin{center}
        \includegraphics[width=0.5\textwidth]{ex2.png}
        \centering
        \captionsetup{justification=centering}
        \caption{\label{fig:ex2}Comparaison entre fonctions de répartition théorique et empirique pour $n = 1000$ réalisations}
    \end{center}
\end{figure}

On voit que nos réalisations semblent conformes a la loi de $X$.

%-------------------------------------------------------------------------------

\section{Exercice 3}\label{sec:ex3}

Etant donné la fonction de répartition de la loi de Cauchy, on génère des
réalisations par la méthode d'inversion:

\begin{equation}
    \begin{split}
        F(x) &= \frac{1}{\pi} arctan\left(\frac{x - x_0}{a}\right) + \frac{1}{2} = y \\
             & arctan\left(\frac{x - x_0}{a}\right) = \left(y - \frac{1}{2}\right) \pi \\
             & \frac{x - x_0}{a} = tan\left(\left(y - \frac{1}{2}\right) \pi \right) \\
             & x = a \, tan\left(\left(y - \frac{1}{2}\right) \pi \right) + x_0
    \end{split}
\end{equation}

On choisit de générer $n = 1000$ réalisations de $X \sim Cauchy(1, 0)$ en
générant $n = 1000$ réalisations $u_i$ de $U \sim \mathcal{U}([0, 1])$ et on
leur applique la transformation
$tan\left(\left(u_i - \frac{1}{2}\right) \pi \right)$ pour obtenir des
réalisations de $X$.

On vérifie notre raisonnement en comparant fonction de répartition théorique (en
noir) et fonction de répartition empirique (en pointillés rouges):

\begin{figure}[H]
    \begin{center}
        \includegraphics[width=0.5\textwidth]{cauchy.png}
        \centering
        \captionsetup{justification=centering}
        \caption{\label{fig:cauchy}Comparaison entre fonctions de répartition théorique et empirique de la loi de Cauchy $x_0 = 0$ et $a = 1$ a partir $n = 1000$ réalisations}
    \end{center}
\end{figure}

On voit que nos réalisations semblent véritablement suivre une loi de Cauchy.

La loi forte des grands nombres nous dit que:

\begin{equation}
    \frac{1}{n} \sum_{i = 1}^{n} x_i \to \mathbb{E}(X) \text{, presque sûrement}
\end{equation}

Pour voir si la loi de Cauchy vérifie la loi forte des grands nombres, on
effectue la moyenne empirique des $x_i$ pour chaque $n$. Si on trace ces
$n = 1000$ moyennes, on obtient:

\begin{figure}[H]
    \begin{center}
        \includegraphics[width=0.5\textwidth]{cauchyLgn.png}
        \centering
        \captionsetup{justification=centering}
        \caption{\label{fig:cauchyLgn}Divergence de la moyenne empirique des réalisations en fonction du nombre de réalisations générées pour $n = 1000$ réalisations}
    \end{center}
\end{figure}

On voit que cette moyenne diverge. En effet, elle "saute" périodiquement, puis
recommence à converger pour se remettre à diverger.

Ceci s'explique par le fait que l'on génère des $x_i$ à l'aide de la fonction
$tan$ qui prend des valeurs extrêmes lorsque $u_i$ se rapproche de 1 ou 0.

En conclusion, la loi de Cauchy ne vérifie pas la loi forte des grands nombres
et n'a donc pas d'espérance.

%-------------------------------------------------------------------------------

\section{Conclusion}\label{sec:conclu}

En conclusion, ce TP nous aura permis de voir des applications de la
génération de nombres aléatoires et, plus particulièrement, l'approximation
d'intégrale lorsque le calcul de celle-ci est difficile analytiquement. On
aurait, d'ailleurs, pu calculer un intervalle de confiance sur cette intégrale
ainsi qu'effectuer des comparaisons avec les méthodes de calculs d'intégrales
usuelles comme, par exemple, celle du point milieu.

Enfin, dans le cadre de ~\ref{sec:ex3}, on aurait pu comparer notre méthode de
génération de réalisations de la loi de Cauchy par inversion avec la méthode
de Box-Müller car cette méthode permet la génération de deux variables
aléatoires $X, Y \sim \mathcal{N}(0, 1)$ et $\frac{X}{Y} \sim Cauchy(0, 1)$.

%-------------------------------------------------------------------------------

\end{multicols}
\end{document}
