% !TEX encoding = IsoLatin
\documentclass{article}
\usepackage[french]{babel}
\usepackage[utf8]{inputenc}
\usepackage[T1]{fontenc}

\usepackage{amsmath}
\usepackage{amssymb}
\usepackage{amsfonts}
\usepackage{graphicx}

\usepackage[hmarginratio=1:1,top=32mm,columnsep=20pt]{geometry}
\usepackage{multirow}
\usepackage{multicol}
\usepackage{abstract}
\usepackage{fancyhdr}
\usepackage{float}

\usepackage[colorlinks=true,linkcolor=red,urlcolor=blue,filecolor=green]{hyperref}

\usepackage{dtklogos}
\usepackage{pbox}
\usepackage{caption}
\usepackage{mathtools}
\usepackage{listings}
\usepackage{mathrsfs}

\pagestyle{fancy}
\fancyhead{}
\fancyfoot{}
\fancyhead[C]{TP A: Modele de Markov cache et l'apprentissage automatique}
\fancyfoot[RO, LE]{\thepage}

\newcommand{\bfx}{\mathbf(x)}
\newcommand{\transp}{^{\mathrm{t}}}

%-------------------------------------------------------------------------------

\title{Compte-rendu Modele de Markov cache et l'apprentissage automatique}

\author{Julien Amalfi, Benjamin Fradet}
\date{\today}

%-------------------------------------------------------------------------------

\begin{document}
\maketitle
\thispagestyle{fancy}

%-------------------------------------------------------------------------------

\begin{abstract}
\end{abstract}

%-------------------------------------------------------------------------------

\begin{multicols}{2}

\section{Introduction}\label{sec:intro}

%-------------------------------------------------------------------------------

\section{Presentation du modele}\label{sec:model}
JULIEN

\section{Presentation des donnees}\label{sec:donnees}
JULIEN

\section{Construction du modele}\label{sec:construct}

Nous avons defini notre ensemble d'etats tel que:

\begin{equation}
    E = {\text{basse pression}, \text{haute pression}}
\end{equation}

et notre ensemble d'observations:

\begin{equation}
    O = {\text{soleil}, \text{nuage}, \text{pluie}, \text{neige}}
\end{equation}

Nous calculons nos matrices d'emissions et de transitions a partir de nos
donnees:

\begin{equation}
    M_{transitions} =
    \begin{pmatrix*}
        0.7099 & 0.2824 \\
        0.1581 & 0.8419
    \end{pmatrix*}
\end{equation}

\begin{equation}
    M_{emissions} =
    \begin{pmatrix*}
        0.1832 & 0.5573 & 0.1145 & 0.1450 \\
        0.2820 & 0.2350 & 0.0427 & 0.4402
    \end{pmatrix*}
\end{equation}

Ceci nous donne le modele suivant:

\begin{figure}[H]
    \begin{center}
        \includegraphics[width=0.5\textwidth]{empiricalHmm.png}
        \centering
        \captionsetup{justification=centering}
        \caption{\label{fig:empiricalHmm}Modele de Markov cache empirique}
    \end{center}
\end{figure}

\section{Exploitation du modele}\label{sec:exploit}

\subsection{Generation d'une sequence d'observations}

A partir du modele precedent, et en particulier des matrices d'emissions et de
transitions ainsi que de la loi initiale, on peut generer une sequence
d'etats/observations.

En effet, on commence par choisir un etat de depart a l'aide de la loi initiale,
puis on cherche une observation en fonction de l'etat choisi a l'aide de la
matrice d'emission. Enfin, on trouve un nouvel etat a l'aide de la matrice de
transitions. On repete les deux dernieres etapes jusqu'a avoir la longueur de la
sequence voulue. Cet algorithme est implemente dans la fonction
\emph{generateHMMSeq}.

A titre d'exemple, avec notre modele, si on genere une sequence de $n = 10$
couples etats/observations, on obtient:

\begin{table}[H]
    \begin{center}
        \centering
        \captionsetup{justification=centering}
        \caption{\label{tab:hmmSeq}Sequence d'etats/observations generes a partir de notre model}
        \begin{tabular}{|c|c|c|}
            \hline
            & etat & observation \\
            \hline
            1 & basse pression & nuage \\
            2 & basse pression & nuage \\
            3 & haute pression & soleil \\
            4 & haute pression & soleil \\
            5 & haute pression & pluie \\
            6 & haute pression & soleil \\
            7 & basse pression & pluie \\
            8 & haute pression & nuage \\
            9 & haute pression & soleil \\
            10 & haute pression & pluie \\
            \hline
        \end{tabular}
    \end{center}
\end{table}

\subsection{Posterior state probabilities}

\subsection{Most probable path}

\subsection{Model learning}

%-------------------------------------------------------------------------------

\section{Conclusion}\label{sec:conclu}

%-------------------------------------------------------------------------------

\end{multicols}
\end{document}
